\documentclass{ctexart}
 \usepackage{geometry}
\usepackage{newtxtext}
\usepackage[dvipsnames,svgnames]{xcolor}
\usepackage{tcolorbox}
\usepackage{amssymb}  
\usepackage{amsmath}  
\usepackage{amsthm}
\usepackage[T1]{fontenc}
\usepackage[CJKbookmarks=true, colorlinks=true]{hyperref}
\newtheorem{question}{问题} % theorem is what you type and Theorem is what to be display
\newtheorem{prop}{命题}[section]
\newtheorem{Theorem}{定理}[section]
\newcommand{\q}[1]{\begin{question}

    #1 \end{question}}
\newcommand{\diff}[2]{\frac{\partial#1}{\partial#2}}
\newcommand{\twodi}[1]{\frac{\partial^2}{\partial #1^2}}
\theoremstyle{definition}
\newtheorem{Definition}{定义}[section]
\newcommand{\itg}[2]{\int^{#2}_{#1}}
\newcommand{\grd}{\nabla}
\geometry{a4paper,centering,scale=0.8}
\def\themytheorem{\arabic{mytheorem}}
% characters 
\newcommand{\der}{\mathrm{der}}
\newcommand{\ga}{\mathfrak{a}}
\newcommand{\gb}{\mathfrak{b}}
\newcommand{\gc}{\mathfrak{c}}
\newcommand{\gp}{\mathfrak{p}}
\newcommand{\gm}{\mathfrak{m}}
\newcommand{\gR}{\mathfrak{R}}
\newcommand{\Z}{\mathbb{Z}}
\newcommand{\Q}{\mathbb{Q}}
\newcommand{\R}{\mathbb{R}}
\newcommand{\C}{\mathbb{C}}
\newcommand{\K}{\mathbb{K}}
\newcommand{\id}{\mathrm{id}}
\newcommand{\N}{\mathbb{N}}
\newcommand{\la}{\langle}
\newcommand{\ra}{\rangle}

\newcommand{\vocab}[1]{{\color{red}{\textit{#1}}}}

% \tcbset{
% defstyle/.style={fonttitle=\bfseries\upshape,
% arc=0mm, colback=yellow!5,
% colframe=yellow!80!black},
% theostyle/.style={fonttitle=\bfseries\upshape, colback=blue!5,colframe=blue!50!black},
% corstyle/.style={fonttitle=\bfseries\upshape,
% colback=green!5,colframe=green!30!black}}
\title{复分析笔记}

\begin{document}
\tableofcontents 

\part{复数}
\section*{数论}
\begin{question} 两个平方和乘积也是一个平方和。 \[(a^2+b^2)(c^2+d^2)=(()^2+()^2)\] 
 \end{question}
\begin{proof}  \end{proof}
\section*{球极投影}
\begin{Definition}{}[扩展复平面] 集合
    \(\mathbb{C}\cup \{\infty \}\) 称为扩展复平面,记为
    \(\mathbb{C}_{\infty}\). 对于$z,\in \mathbb{C}_{\infty}$, 定义度量 \(d(z,\infty)=d(\infty,z)=\infty\).  
\end{Definition}
从扩展平面到三维球面\(S\)有映射\(f\)定义如下
\begin{align}
    f(z)=\bigg(\frac{z+\bar{z}}{|z|^2+1} , \frac{-i(z-\bar{z})}{|z|^2+1}, \frac{|z|^2-1}{|z|^2+1}\bigg)
\end{align}


我们来证明球极投影是一个连续的双射,并且其逆映射也连续。

\begin{prop}{}
    \(f^{-1}\colon S\to \mathbb{C}_{\infty}\) 存在,而且对任意\((x_1,x_2,x_3)\in S\), 有\[f^{-1}(x_1,x_2,x_3)=\frac{x_1+ix_2}{1-x_3}\]. 
\end{prop}
把\(\mathbb{C}_{\infty}\)看作是\(\mathbb{R^2}_{\infty}\),也可以写 \[ f^{-1}(x_1,x_2,x_3)=\bigl( \frac{x_1}{1-x_3},\frac{x_2}{1-x_3} \bigl). \]
\begin{prop}\hypertarget{02}{}
    \(g\colon S\to \mathbb{R}\colon (x_1,x_2,x_3) \mapsto \bigl( \frac{x_1}{1-x_3},\frac{x_2}{1-x_3} \bigl) \) 是连续函数。
    
\end{prop}   
要证明定理 \hyperlink{02}{0.2},我们有下面更一般的命题。
\begin{prop}\hypertarget{03}{}
    函数 \(f\colon Z\to X\times Y\) 连续当,且仅当其分量函数\[f_1\colon Z\to X\] 和 \[f_2\colon Z\to Y\] 都连续。 
\end{prop}
忽略对这个一般命题的证明,直接用它来证明

\begin{proof}
    对于\(x=(x_1,x_2,x_3)\), 设\(S\to \mathbb{R}\)的函数 \(g_1(x)=\frac{x_1}{1-x_3}\), \(g_2(x)=\frac{x_2}{1-x_3}\). 那么\(f^{-1}(x)=(g_1(x),g_2(x))\).
    \begin{itemize}
        \item \(g_1\),\(g_2\)都是连续函数.
        
        这是因为
            \(g_1\)可以分解为 \((x_1,x_2,x_3)\mapsto (x_1,1-x_3) \mapsto \frac{x_1}{1-x_3} \). \(g_2\)类似。
        \item  由于 \(g_1,g_2\)都连续,所以,根据 \hyperlink{03}{命题0.3} \(f^{-1}\)也连续。
    \end{itemize}
\end{proof}

\section{The logarithm and inverse functions}

Branches.
\subsection{Inverse \(\sin\) functions}

$w=\sin^{-1}(z)$ then $z=sin(w)=\frac{e^{iw}-e^{-iw}}{2i}$ from which 

\subsection{Inverse Hyperbolic functions}
\[z=\frac{e^{w}-e^{-w}}{e^{w}+e^{-w}}.\] from which 
\[e^{2w} =  \] 
Solve for $e^{w}$ and take $\ln$ .

Note that $\ln(x)$ has infinite many branches while $\sqrt[5]{x}$ only has 5 branches.


\section{Limits}

\section{Cauchy Riemann Equation}

Cauchy Riemann Equation not only provides a way to check whether a complex variabled function is differentiable, but also is a tool to compute the derivative. It also connects the differential of multivariable functions and complex variable fucntions.
\section{Harmonics}
\begin{Definition}
Harmonics if \(\varphi\) is twice differentiable on \(\R\) with continuous second derivative.
\end{Definition}

If \(\frac{\partial^2}{\partial^2} \varphi + \frac{\partial^2}{\partial y^2}\varphi =0 \).  Usually written as \(\partial^2 \varphi=0\). 

Examples
\begin{enumerate}
\item \(\varphi(x,y)=x^2y^2\). Then \(\twodi{x} \varphi(x)  +\twodi{y} \varphi = 2+(-2)=0.\)
\item \[\varphi(x,y)=\sin(x)\cosh(y).\] Then \[\diff{ \cos(x)\cosh(y)}{x} \to \sin(x)\cosh(y)\]
\end{enumerate}

\begin{Theorem}
Holomorphic implies real and imaginary part harmonic.
\end{Theorem}

Given that we just need to check that \(u\) and \(v\) satisfies \(\grd^2v=0\) and \(\grd^2v=0\). Since \(u\) and \(v\) are the imaginary part of the holomorphic function, they satisfy the cauchy rieman equation. 
Use cauhy-riman equation to show \(u\) and \(v\) are harmonic.

\begin{Theorem}[Reconstruction theorem]
\(R\in \R^2\) open set simply-connected set. \(u\colon R\to R\) a harmonic function. Then there is a harmonic function \(v\colon R\to R\) so that \(f(x+yi)=u(x,y)+iv(x,y)\) is a holomophic function on \(R\). 
\end{Theorem}

The function \(v\) is called the \vocab{harmonic conjugate } of \(u\) and it is unique up to the addition of a constant function. 

\begin{proof} By the cauchy-riemenn theorem for a function \(f(x+iy)=u(x,y)+iv(x,y)\) to be holomorphic, we need \(u\) and \(v\) to satisfy the cauchy riemann equations. \[\diff{v}{x}=-\diff{u}{y} \] and \[\diff{v}{y}=\diff{u}{x}\]
\(G\colon R\to \R^2\) be te vector field with component functions \(G_1=\diff{u}{y}\) and \(G_2=\diff{u}{x}\). \(G=(-\diff{u}{y}, \diff{u}{x})\).
So C-R equation can be written as \(\grd v= (-\frac{\partial u}{\partial y}, \frac{\partial u}{\partial x}) = G\). 

% Since \(Curl (G)= \frac{\partial G_2}{\partial x}-\frac{\partial G_1}{\partial y=\frac{\partial }{\partial x} \frac{\partial u}{\partial y}- \frac{\partial}{\partia x} \frac{\partial u }{\parital y}= -(\frac{\partial^2 u}{\partial x^2}+\frac{\partial^2 v}{\partial x})=0\)

So \(v\) exists. 

To show \(F(x,y)=(u(x,y),v(x,y))\) is differentiable on \(R\) it is enough to show that the first partial derivative is \(u\) and \(v\) exist and are continuous on \(R\). 

Since \(u\) is harmonic, its second derivatives exist, and therefore the first derivative exist and are continuous. But, by the cr equatons, the first derivatives of \(v\) are (up to sign) the first derivative of \(u\). Hence the first derivatives of \(v\) are also continuous. 

THerefore, \(F\) is multivariable differentiable on \(R\) and so by the C-R theorem, \(f\) is holomophic on \(R\).
\end{proof}
\end{document}

