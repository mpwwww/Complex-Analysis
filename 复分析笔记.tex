\documentclass[font=windows]{ctexart}
 \usepackage{geometry}
\usepackage{newtxtext}
\usepackage[dvipsnames,svgnames]{xcolor}
\usepackage{tcolorbox}
\usepackage{amssymb}  
\usepackage{amsmath}  
\usepackage{amsthm}
\usepackage[T1]{fontenc}
\usepackage[CJKbookmarks=true, colorlinks=true]{hyperref}
\newtheorem{question}{问题} % theorem is what you type and Theorem is what to be display
\newtheorem{prop}{命题}[section]
\newtheorem{Theorem}{定理}[section]
\newcommand{\q}[1]{\begin{question}

    #1 \end{question}}
\theoremstyle{definition}
\newtheorem{Definition}{定义}[section]
\newcommand{\itg}[2]{\int^{#2}_{#1}}
\geometry{a4paper,centering,scale=0.8}
\def\themytheorem{\arabic{mytheorem}}
% \tcbset{
% defstyle/.style={fonttitle=\bfseries\upshape,
% arc=0mm, colback=yellow!5,
% colframe=yellow!80!black},
% theostyle/.style={fonttitle=\bfseries\upshape, colback=blue!5,colframe=blue!50!black},
% corstyle/.style={fonttitle=\bfseries\upshape,
% colback=green!5,colframe=green!30!black}}
\title{复分析笔记}

\begin{document}
\tableofcontents 

\part{复数}
\section*{数论}
\begin{question} 对于$b^2$, $b\in \mathbb{Z^{+}}$, 找出$c,d\in \mathbb{Z}^{+}$使得 $b^2=c^2+d^2$. 
 \end{question}
\begin{proof}  \end{proof}
\section*{球极投影}
\begin{Definition}{}[扩展复平面] 集合
    \(\mathbb{C}\cup \{\infty \}\) 称为扩展复平面,记为
    \(\mathbb{C}_{\infty}\). 对于$z,\in \mathbb{C}_{\infty}$, 定义度量 \(d(z,\infty)=d(\infty,z)=\infty\).  
\end{Definition}
从扩展平面到三维球面\(S\)有映射\(f\)定义如下
\begin{align}
    f(z)=\bigg(\frac{z+\bar{z}}{|z|^2+1} , \frac{-i(z-\bar{z})}{|z|^2+1}, \frac{|z|^2-1}{|z|^2+1}\bigg)
\end{align}


我们来证明球极投影是一个连续的双射,并且其逆映射也连续。

\begin{prop}{}
    \(f^{-1}\colon S\to \mathbb{C}_{\infty}\) 存在,而且对任意\((x_1,x_2,x_3)\in S\), 有\[f^{-1}(x_1,x_2,x_3)=\frac{x_1+ix_2}{1-x_3}\]. 
\end{prop}
把\(\mathbb{C}_{\infty}\)看作是\(\mathbb{R^2}_{\infty}\),也可以写 \[ f^{-1}(x_1,x_2,x_3)=\bigl( \frac{x_1}{1-x_3},\frac{x_2}{1-x_3} \bigl). \]
\begin{prop}\hypertarget{02}{}
    \(g\colon S\to \mathbb{R}\colon (x_1,x_2,x_3) \mapsto \bigl( \frac{x_1}{1-x_3},\frac{x_2}{1-x_3} \bigl) \) 是连续函数。
    
\end{prop}   
要证明定理 \hyperlink{02}{0.2},我们有下面更一般的命题。
\begin{prop}\hypertarget{03}{}
    函数 \(f\colon Z\to X\times Y\) 连续当,且仅当其分量函数\[f_1\colon Z\to X\] 和 \[f_2\colon Z\to Y\] 都连续。 
\end{prop}
忽略对这个一般命题的证明,直接用它来证明

\begin{proof}
    对于\(x=(x_1,x_2,x_3)\), 设\(S\to \mathbb{R}\)的函数 \(g_1(x)=\frac{x_1}{1-x_3}\), \(g_2(x)=\frac{x_2}{1-x_3}\). 那么\(f^{-1}(x)=(g_1(x),g_2(x))\).
    \begin{itemize}
        \item \(g_1\),\(g_2\)都是连续函数.
        
        这是因为
            \(g_1\)可以分解为 \((x_1,x_2,x_3)\mapsto (x_1,1-x_3) \mapsto \frac{x_1}{1-x_3} \). \(g_2\)类似。
        \item  由于 \(g_1,g_2\)都连续,所以,根据 \hyperlink{03}{命题0.3} \(f^{-1}\)也连续。
    \end{itemize}
\end{proof}
\end{document}

